\documentclass[a4]{beamer}
\usepackage{amssymb}
\usepackage{graphicx}
\usepackage{subfigure}
\usepackage{newlfont}
\usepackage{amsmath,amsthm,amsfonts}
%\usepackage{beamerthemesplit}
\usepackage{pgf,pgfarrows,pgfnodes,pgfautomata,pgfheaps,pgfshade}
\usepackage{mathptmx}  % Font Family
\usepackage{helvet}   % Font Family
\usepackage{color}

\mode<presentation> {
 \usetheme{Default} % was Frankfurt
 \useinnertheme{rounded}
 \useoutertheme{infolines}
 \usefonttheme{serif}
 %\usecolortheme{wolverine}
% \usecolortheme{rose}
\usefonttheme{structurebold}
}

\setbeamercovered{dynamic}

\title[MA4413]{Statistics for Computing \\ {\normalsize MA4413 Lecture 4B}}
\author[Kevin O'Brien]{Kevin O'Brien \\ {\scriptsize Kevin.obrien@ul.ie}}
\date{Autumn Semester 2012}
\institute[Maths \& Stats]{Dept. of Mathematics \& Statistics, \\ University \textit{of} Limerick}

\renewcommand{\arraystretch}{1.5}

\begin{document}

\begin{frame}
\titlepage
\end{frame}

%---------------------------------------------------------------%
\frame{
\frametitle{Today's Class}
\begin{itemize}
\item Review of Discrete Probability Distributions
\item \texttt{R} Implementation
\item A Few Examples
\item Introduction to Continuous Probability Distributions
\item The Uniform Distribution
\item The Exponential Distribution
\end{itemize}
}
%---------------------------------------------------------------------------%
\frame{
\frametitle{Discrete Probability Distributions}
Three main distributions
\begin{itemize}
\item Binomial Distribution
\item Poisson Distribution
\item Geometric Distribution (mentioned, but not as important as the other two.)
\end{itemize}
}
%---------------------------------------------------------------------------%
\frame{
\frametitle{Binomial Probability Distribution}
Important Points:
\begin{itemize}
\item The experiment is a series of $n$ independent trials.
\item Two possible outcomes from each trial: a success and a failure.
\item The probability of success (i.e. $p$) is constant. 
\item A binomial random variable can be written as 
\[ X \sim B(n,p) \]
\end{itemize}
}
%---------------------------------------------------------------------------%
\frame{
\frametitle{Poisson Probability Distribution}
Important Points:
\begin{itemize}
\item This distribution is concerned with the number of occurrences per unit space.
\item Unit space can mean a unit length, a unit area, a unit volume or a unit period of time.
\item We will concern ourselves with unit time periods mostly.
\item A Poisson random variable can be written as
\[ X \sim Pois(m) \]
\item The Poisson distribution can be used to approximate the binomial distribution under certain conditions.
\end{itemize}
}
%---------------------------------------------------------------------------%
\frame{
\frametitle{PDFs and CDFs}
Important Points:
\begin{itemize}
\item The probability density function (PDF) is the probability of a random variable taking a specific value i.e. $P(X = k)$
\item The appropriate \texttt{R} functions are \texttt{dbinom} and \texttt{dpois}
\item The cumulative distribution function (CDF) is the probability of a random variable not exceeding a specific value i.e. $P(X \leq k)$
\item The appropriate \texttt{R} functions are \texttt{pbinom} and \texttt{ppois}     
\end{itemize}
}

%---------------------------------------------------------------------------%
\frame{
\frametitle{Geometric Probability Distribution}
Important Points:
\begin{itemize}
\item This distribution is closely related to the binomial distribution. 
\item This distribution described the number of failures that occur before the first success, when the probability of success is $p$.
\item The relevant \texttt{R} functions are \texttt{dgeom} and \texttt{pgeom}.
\end{itemize}
}

%---------------------------------------------------------------------------%
\begin{frame}[fragile]
\frametitle{Geometric Probability Distribution : Example}

If the probability of inserting a USB correctly is $0.40$, what is the probability of successfully doing so on the second attempt.
\bigskip.
In essence we have one failure, then one success, and these are independent events. So the probability the second attempt will be successful is $0.6 \times 0.4$. The probability that we are successful on the first attempt (i.e. no failures beforehand) is 0.4

\begin{verbatim}
> dgeom(0,prob=0.4)
[1] 0.4
> dgeom(1,prob=0.4)
[1] 0.24
> dgeom(2,prob=0.4)
[1] 0.144
\end{verbatim}

\end{frame}




%---------------------------------------------------------------------------%
\frame{
\frametitle{Cumulative Binomial Probability Distribution Tables}

\begin{itemize}
\item Consider a binomial experiment with $n = 20$ and $p = 0.50$.

\item First we have to find which section of the tables which tabulates the correspond binomial probabilities for $n = 20$ and $p = 0.50$

\item Lets use the tables to compute $P(X \geq 10)$

\item Remark $P(X \geq 10) = P(X=10) + P(X=11) + P(X=12)+ \ldots +  P(X=20)$

\end{itemize}

}
%---------------------------------------------------------------------------%
\frame{
\frametitle{Cumulative Binomial Probability Distribution Tables}

Section of tables for $n=20$ and $p = (0.10,0.15, \ldots, 0.50)$
\begin{center}
\begin{tabular}{|r||c|c|c|}
  \hline
  % after \\: \hline or \cline{col1-col2} \cline{col3-col4} ...
  p= & \ldots & 0.45 & \textbf{0.50} \\ \hline \hline
  r=0 & \ldots & 1.0000 & 1.0000 \\
  1 & \ldots & 1.0000 & 1.0000 \\
  2 & \ldots & 0.9999 & 1.0000 \\
  \ldots & \ldots & \ldots & \ldots \\
  9  & \ldots & 0.5857 & 0.7483 \\
  \textbf{10} & \ldots & 0.4086 & \alert{\textbf{0.5881}} \\
  11 & \ldots & 0.2493 & 0.4119 \\
  \ldots & \ldots & \ldots & \ldots \\
  \hline
\end{tabular}
\end{center}

}

%---------------------------------------------------------------------------%
\frame{
\frametitle{Cumulative Binomial Probability Distribution Tables}

\begin{itemize}
\item Consider a binomial experiment with $n = 25$ and $p = 0.50$.
\item Suppose we wish to compute $P(X \geq 10)$

\item Is there a section of the tables which tabulates the correspond binomial probabilities for $n = 25$ and $p = 0.50$

\item No! We would not be able to use the Murdoch Barnes tables to compute this.


\item Of course, we could more detailed set of tables, but that will not part of this module. For all problems using \textbf{cumulative} probabilities will be restricted to those that can be solved using the Murdoch Barnes tables.

\end{itemize}
}



%---------------------------------------------------------------------------%
\frame{
\frametitle{Cumulative Binomial Probability Distribution Tables}

\begin{itemize}
\item Consider a binomial experiment with $n = 20$ and $p = 0.50$.
\item Suppose we wish to compute the probability of 9 successes or less, i.e. $P(X \leq 9)$
\item The tables does not tabulate probabilities in such a way. Recall that it tabulates probabilities for $r$ successes \textbf{or more}.
\item However, we can still use the table to compute the desired probability.
\item Consider the sample space for the number of successes. There can be between 0 and 20 successes.
\Large
\[ S = \{0,1,2,3,\ldots,8,9,10,11,12 \ldots,19,20\}
\]
\end{itemize}
}
%---------------------------------------------------------------------------%
\frame{
\frametitle{Cumulative Binomial Probability Distribution Tables}

\begin{itemize}
\item What are the sample points for the event where the number of success is less than or equal to 9? (Lets call this event $A$.)
\[ A = \{0,1,2,3,\ldots,8,9\}
\]
\item What are the sample points for the \textbf{\emph{complement event}}. (Lets call this event $A^c$).
\[ A^c = \{10,11,12 \ldots,19,20\}
\]
\item Can we compute the probability of event $A^c$? \\ Yes, it is $P(X\geq 10)$
\item The solution is therefore \[P(X\leq 9) = 1- P(X \geq 10) = 1 - 0.5881 = 0.4119\]
\end{itemize}
}

%---------------------------------------------------------------------------%
\frame{
\frametitle{Cumulative Binomial Probability Distribution Tables}

\begin{itemize}
\item What is the probability that the number of successes is exactly 10 (with  $n=20$, $p=0.50$)?

\item i.e. $P(X = 10) = ?$

\item Again, the tables does not tabulate probabilities in such a way.

\item It would also be quite cumbersome to compute $P(X=10)$ using the binomial probability formula.

\item In the previous question, we found out the probability of 10 or more successes.
\end{itemize}
}

%---------------------------------------------------------------------------%
\frame{
\frametitle{Cumulative Binomial Probability Distribution Tables}
\normalsize
\begin{itemize}


\item Recall that $P(X \geq 10) = P(X=10) + P(X=11) + P(X=12)+ \ldots +  P(X=20)$.
\bigskip
\item Equivalently $P(X \geq 11) = P(X=11) + P(X=12)+ \ldots +  P(X=20)$.
\bigskip
\item Therefore $P(X \geq 10) = P(X=10) + P(X \geq 11)$.
\bigskip
\item Re-arranging this expression we get $P(X=10)  = P(X \geq 10) -  P(X \geq 11)$.

\bigskip
\item From the tables $P(X \geq 11) = 0.4119$.
\bigskip
\item Therefore $P(X=10) = 0.5881 -  0.4119)  = 0.1762$.
\end{itemize}
}

%---------------------------------------------------------------------------%
\frame{
\frametitle{Cumulative Binomial Probability Distribution Tables}

Section of tables for $n=20$ and $p = (0.10,0.15, \ldots, 0.50)$
\begin{center}
\begin{tabular}{|r||c|c|c|}
  \hline
  % after \\: \hline or \cline{col1-col2} \cline{col3-col4} ...
  p= & \ldots & 0.45 & \textbf{0.50} \\ \hline \hline
  r=0 & \ldots & 1.0000 & 1.0000 \\
  1 & \ldots & 1.0000 & 1.0000 \\
  2 & \ldots & 0.9999 & 1.0000 \\
  \ldots & \ldots & \ldots & \ldots \\

  10 & \ldots & 0.4086 & 0.5881 \\
  \textbf{11} & \ldots & 0.2493 & \alert{\textbf{0.4119}} \\
   12  & \ldots & 0.1308 & 0.2517 \\
  \ldots & \ldots & \ldots & \ldots \\
  \hline
\end{tabular}
\end{center}

}


%---------------------------------------------------------------------------%
\frame{
\frametitle{Cumulative Binomial Probability Distribution Tables}
The vice-president of a computer firm has reviewed the records of the firm�s personnel and has found that 70\% of the employees read a well known industry magazine ``The IT Journal". \\ \bigskip
If the vice-president was to choose 10 employees at random, what is the probability that the number of these employees who do not read the ``IT Journal" is the following?
\normalsize
\begin{itemize}
\item [1] At least five.
\item [2] Between four and eight, inclusive.
\item [3] No more than seven.
% \item [4] What are the mean and variance of this distribution?
\end{itemize}
}

%---------------------------------------------------------------------------%
\frame{
\frametitle{Cumulative Binomial Probability Distribution Tables}
\textbf{Solution :}

\begin{itemize}
\item
Firstly, identify the probability distribution to be used?
    \begin{itemize}
    \item
    Answer: the binomial distribution
    \end{itemize}
\item We are given the number of trials ( `` choose 10 employees")

\item We are given a definition of a ``success", which is finding an employee that did NOT reads the journal.

\item We are given the probability of such a success : 30\%  or 0.30

\item So our binomial parameters are n= 10 and p = 0.30

\item Let's use the Murdoch Barnes Table 1 and find the relevant columns.

\end{itemize}
}



%---------------------------------------------------------------------------%
\frame{
\frametitle{Cumulative Binomial Probability Distribution Tables}
\normalsize
Section of tables for $n=10$ and $p = (0.10,0.15,\ldots,0.30, \ldots, 0.50)$\\
\bigskip
Part 1:  compute $P(X \geq 5)$ \\ \textbf{Answer}: $P(X \geq 5)  = 0.1503$

\begin{center}
\begin{tabular}{|r||c|c|c|c|}
  \hline
  % after \\: \hline or \cline{col1-col2} \cline{col3-col4} ...
  p= & \ldots & 0.25 & \textbf{0.30} &  \ldots\\ \hline \hline
  r=0 & \ldots & 1.0000 & 1.0000 & \ldots\\
  1 & \ldots & 0.9437 & 0.9718 & \ldots \\
  2 & \ldots & 0.7560 & 0.8507 &\ldots \\
  \ldots & \ldots & \ldots & \ldots&\ldots \\
  \textbf{5} & \ldots & 0.0781 & \alert{\textbf{0.1503}} &\ldots \\
 \ldots & \ldots & \ldots & \ldots&\ldots \\
  \hline
\end{tabular}
\end{center}

}

%---------------------------------------------------------------------------%
\frame{
\frametitle{Cumulative Binomial Probability Distribution Tables}
\normalsize
Part 2: Compute $P(4 \leq X \leq 8)$  \\ \bigskip
\begin{itemize}
\item $P(4 \leq X \leq 8) = P(X = 4) + P(X = 5) + P(X = 6) + P(X = 7) + P(X=8)$
\bigskip
\item $P(X \geq 4) = P(X = 4) +  \ldots + P(X=8) + P(X=9) + P(X=10)$
\bigskip
\item $P(X \geq 9) = P(X = 9) + P(X = 10)$
\bigskip
\item $P(X \geq 4) = P(4 \leq X \leq 8) + P(X \geq 9)$
\bigskip
\item Re-arranging $P(4 \leq X \leq 8) = P(X \geq 4) - P(X \geq 9)$
\end{itemize}
}
%---------------------------------------------------------------------------%
\frame{
\frametitle{Cumulative Binomial Probability Distribution Tables}
\normalsize

Part 2:  compute $P(4 \leq X \leq 8)$ \\ \textbf{Answer}: $P(4 \leq X \leq 8)  = 0.3504-0.0001 = 0.3503$

\begin{center}
\begin{tabular}{|r||c|c|c|c|}
  \hline
  % after \\: \hline or \cline{col1-col2} \cline{col3-col4} ...
  p= & \ldots & 0.25 & \textbf{0.30} &  \ldots\\ \hline \hline
  r=0 & \ldots & 1.0000 & 1.0000 & \ldots\\
   1 & \ldots & 0.9437 & 0.9718 & \ldots \\
  \ldots & \ldots & \ldots & \ldots&\ldots \\
  \textbf{4} & \ldots & 0.2241 & \alert{\textbf{0.3504}} &\ldots \\
 \ldots & \ldots & \ldots & \ldots&\ldots \\
   \textbf{9} & \ldots &  & \alert{\textbf{0.0001}} &\ldots \\
 \ldots & \ldots & \ldots & \ldots&\ldots \\
  \hline
\end{tabular}
\end{center}

}


%---------------------------------------------------------------------------%
\frame{
\frametitle{Cumulative Binomial Probability Distribution Tables}
\normalsize
Part 3: Compute $P(X \leq 7)$
\begin{itemize}
\item From before, this is the complement of 8 or more successes \item i.e. $P(X \leq 7) = 1 - P(X \geq 8)$
\item Determine $P(X \geq 8)$ and subtract that value from 1.
\end{itemize}
}


%---------------------------------------------------------------------------%
\frame{
\frametitle{Cumulative Binomial Probability Distribution Tables}
\normalsize

Part 3:  compute $P(X \leq 7)$ \\ \textbf{Answer}: $P(X \leq 7)  = 1 - 0.0016 = 0.9984$

\begin{center}
\begin{tabular}{|r||c|c|c|c|}
  \hline
  % after \\: \hline or \cline{col1-col2} \cline{col3-col4} ...
  p= & \ldots & 0.25 & \textbf{0.30} &  \ldots\\ \hline \hline
  r=0 & \ldots & 1.0000 & 1.0000 & \ldots\\
   1 & \ldots & 0.9437 & 0.9718 & \ldots \\

 \ldots & \ldots & \ldots & \ldots&\ldots \\
    7 & \ldots & 0.0.0035 & 0.0106 & \ldots \\
   \textbf{8} & \ldots & 0.0004 & \alert{\textbf{0.0016}} &\ldots \\
 \ldots & \ldots & \ldots & \ldots&\ldots \\
  \hline
\end{tabular}
\end{center}

}

%---------------------------------------------------------------------------%
\frame{
\frametitle{Cumulative Poisson Probability Distribution Tables}

\begin{itemize}
\item The Murdoch Barnes Table set 2 (``cumulative Poisson probabilities") tabulates cumulative values for Poisson probabilities.


\item For some value $r$, these tables give the probability for $r$ or more occurrences (i.e $P(X \geq r)$) for some value $m$, the expected number of occurences per unit period.

\item To use the tables, the correct column for $m$ must be found.
\item The cumulative Poisson tables is easier to use, compared to the cumulative binomial tables.
\item All of the problem solving approaches we learned for the cumulative binomial tables, also apply for the cumulative Poisson tables.
\end{itemize}
}

%---------------------------------------------------------------------------%
\frame{
\frametitle{Cumulative Poisson Probability Distribution Tables}

\begin{center}
\begin{tabular}{|r||c|c|c|}
  \hline
  % after \\: \hline or \cline{col1-col2} \cline{col3-col4} ...
  m= & \ldots & 0.40 & 0.50 \\ \hline \hline
  r=0 & \ldots & 1.0000 & 1.0000 \\
  1 & \ldots & 0.3297 & 0.3935 \\
  2 & \ldots & 0.0616 & 0.0902 \\
  3 & \ldots & 0.0079 & 0.0144 \\
  \ldots & \ldots & \ldots & \ldots \\
  \hline
\end{tabular}
\end{center}

}

%---------------------------------------------------------------------------%
\frame{
\frametitle{Cumulative Poisson Probability Distribution Tables}

\begin{itemize}
\item Suppose that for a Poisson random variable that mean number of occurrences per minute was 0.5.
\item Compute the probability that there will more than one occurrence in a given minute.
\item Solution : $P(X \geq 1) = 0.3935$
\end{itemize}

}

%---------------------------------------------------------------------------%
\frame{
\frametitle{Cumulative Poisson Probability Distribution Tables}

$ P(X \geq 1)  = 0.3935$
\begin{center}
\begin{tabular}{|r||c|c|c|}
  \hline
  % after \\: \hline or \cline{col1-col2} \cline{col3-col4} ...
  m= & \ldots & 0.40 & \textbf{0.50} \\ \hline \hline
  r=0 & \ldots & 1.0000 & 1.0000 \\
  \textbf{1} & \ldots & 0.3297 & \textbf{\alert{0.3935}} \\
  2 & \ldots & 0.0616 & 0.0902 \\
  3 & \ldots & 0.0079 & 0.0144 \\
  \ldots & \ldots & \ldots & \ldots \\
  \hline
\end{tabular}
\end{center}

}
\end{document}

