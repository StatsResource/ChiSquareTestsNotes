


\subsection{Hypothesis Test}

Step 1:  Formally write out null and alternative hypothesis

\begin{itemize}
	\item Gender and Choice of College coure are independent of each other.
	
	\item Gender and Choice of College coure are not independent of each other.
\end{itemize}

Step 2: Test Statistic

We use a special Test statistic for this test.

For each of the six subgroups, perform the following calculation.

\[\frac{(n_{ij}-e_{ij})^2}{e_{ij}}\]

\begin{itemize}
	\item nij : observed number for subgroup
	\item eij : expected number for subgroup
\end{itemize}


Add up all these terms.

\[T=\frac{(20-15)^2 }{15} + \frac{(10-15)^2 }{15} + \frac{(10-15)^2 }{15} + \frac{(20-15)^2 }{15} + \frac{(30-30)^2 }{30} + \frac{(30-30)^2 }{30}\]

\[
T= 1.667 + 1.667 +1.667 +1.667 +0 +0 = 6.6667 
\]



Step 3: Test Statistic

\begin{itemize}
	\item Murdoch Barnes Table 8
	\item Significance level is 5%
	\item Number of tails is 2
	\item degrees of freedom = (2-1)(3-1) = 1 $\times$ 2 = 2
\end{itemize}



Critical value is 7.378

\textbf{Step 4 Decision Rule}

Is the Test statistic greater than the Critical value

is 6.6667 > 7.378

No! We fail to reject the null hypothesis.

We do not have enough evidence to say that there is a relationship between gender and college courses.









%=========================================================%

\section{Chi Square}


$p_{i}$ = Expected proportion for digit $i$.

For this test we used the chi-squared test statistic which is given by:
\begin{equation}
X^2 = \sum_{i=1}^{n} {(O_i - E_i)^2 \over E_i}
\end{equation}


The Chi Square test tests a null hypothesis stating that the frequency distribution of certain events observed in a sample is consistent with a particular theoretical distribution. The events considered must be mutually exclusive and have total probability 1. A common case for this is where the events each cover an outcome of a categorical variable.

\begin{itemize}
	\item $X^2$ = the test statistic that asymptotically approaches a $\chi^2$ distribution.
	\item $O_i$ = an observed frequency;
	\item $E_i$ = an expected (theoretical) frequency, asserted by the null hypothesis;
	\item $n $  = the number of possible outcomes of each event.
\end{itemize}

The chi-square statistic can then be used to calculate ap-value by comparing the value of the statistic to a chi-square distribution. The number of degrees of freedom is equal to the number of cells ``n'', minus the reduction in degrees of freedom, ``p''.




		%----------------------------------------------------------------------------------------------%
			
			\chapter{13. ChiSquare}
			
			
			Question 4b  
			Chi-square test for independence and Probability [10 Marks]
			
			This question consists of two parts
			
			Basic Probability
			Chi Square Hypothesis test
			
			Lets look at the table first
			
			Important points from the table
			
			1) There are 120 students , 60 male and 60 female
			
			2) There are three programmes
			
			1) Maths - with room for 30 students
			2) Equine studies - with room for 30 students
			3) Chemistry - with room for 60 students.
			
			Let us assume that there a male and female student are equally likely to enter each program. (This is the null hypothesis).
			
			We would expect the each program to have the following compositions.
			
			1) Maths - would have 15 male and 15 female students (each year on average)
			2) Equine studies - would have 15 male and 15 female students (each year on average)
			3) Chemistry -would have 30 male and 30 female students (each year on average)
			
			
			Probability
			Observed Values


\begin{center}
	\begin{tabular}{cccc}
	&	Male	&	Female	&	Sum	\\ \hline
	Maths	&	20	&	10	&	30	\\ \hline
	Eq. Studies	&	10	&	20	&	30	\\ \hline
	Chemistry	&	30	&	30	&	60	\\ \hline
	Sum	&	60	&	60	&	120	\\ \hline

	\end{tabular}
\end{center}
			
			

			
			What is the probability that a randomly chosen person from the sample is an equine student?
			
			There are 120 students altogether, 30 of those are equine science students.
			
			\[P(Eq) = \frac{30}{120}= 0.25\]
			
			Given that a student is female, what is the probability that that she is an equine science student.
			\[P(Eq |F) = \frac{P(Eq \mbox{and} F)}{P(F)}= \frac{20/120}{60/120}= 0.33\]


			
			

\end{document}
